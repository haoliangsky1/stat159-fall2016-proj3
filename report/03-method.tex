\documentclass{article}
\title{Stat159 Project 3: Add a title}
\author{Liang Hao, Bret Hart, Andrew Shibata, Gary Nguyen}
\date{\today}

\usepackage{Sweave}
\begin{document}
\Sconcordance{concordance:03-method.tex:03-method.Rnw:%
1 15 1 50 0}


\maketitle
\section{Method}

In this section, we continue the discussion on how we decided upon the composite "quality" metric for high-quality, low-cost education for at-risk students. We later move on to their weighting and actual method implementation in the shiny app. 

One of the most common reasons students cite in choosing to go to college is the expansion of employment opportunities. This is of even greater intuitive importance to poorer students, who absolutely cannot afford to go to an expensive school that may leave them in permanent debt. To that end, data on the earnings and employment prospects of former students can provide key information. To measure the labor market outcome of individuals attending institutions of higher education, data on cohorts of federally aided students were linked with earnings data from de-identified tax records and reported back at the aggregate, institutional level. This dataset, however, is separate from the main College Scorecard dataset, and is included in a different .csv, *Most-Recent-Cohorts-Treasury-Elements.csv* As mentioned earlier, although these datasets were separate at the start, we merge the two to allow for easy access to pertinent information that may only be contained in one of the datasets. Obvious data of interest only within this dataset are median earnings and median debt after graduation.

So - we choose a set of columns that we feel, in aggregate, can represent in some small, objective measure, a school's quality - now, how do we actually create this score in practice? First, we standardize all of the included variables within their respective columns, then take z-scores, and use these z-scores in place of the actual values within each column. Thus, the average of any statistic will be given a score close to 0, with scores above given a more positive score, and scores below a more negative score. Then, we weight each metric with deliberation, experimentation, and consultation of literature. Importantly, some data points are actually desirable when substantially *below* the average, such as net price or median debt, while others are obviously better if higher, such as median earnings or completion rate. Thus, we actually weight some pieces positively and other negatively in regards to the sign of their z-scores. Finally, these scores will alter and adjust both weighting and actual information as students enter and change their inputted answers. 

Eventually, we decided upon the following for our data included in the response and their weights:


1 - Net Price, stratified by income. There are 5 income brackets within the dataset, so dependent upon which income bracket a student inputs, the score is recalculated to consider the relative z-score of their projected respective net cost of attendance. For students below a certain income level, we actually want to more heavily reward schools with very negative z-scores, as these schools, in totality, are anomalously cheap when compared to their counterparts. We decide that for lower income students, this is of more significant importance, as this is the net price of attending AFTER financial aid - and these are students who need to know exactly how much attending could cost. Thus, the columns the information is drawn from and the weights themselves are dynamically adjusted based on user input.

